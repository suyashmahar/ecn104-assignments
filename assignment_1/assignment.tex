\documentclass[a4paper,10pt]{article}
 
\usepackage[T1]{fontenc}
\usepackage[utf8]{inputenc}
\usepackage{graphicx}
\usepackage{xcolor}
\usepackage{caption}
\usepackage{listings}
\usepackage{wrapfig} 

\renewcommand\familydefault{\sfdefault} 
\usepackage{tgheros}
\usepackage[defaultmono]{droidmono}

\usepackage{amsmath,amssymb,amsthm,textcomp}
\usepackage{enumerate}
\usepackage{multicol}
\usepackage{tikz}

\usepackage{geometry}
\usepackage{trace}
\usepackage{tcolorbox}
\tcbuselibrary{listings,skins} % For lstlisting

\geometry{total={210mm,297mm},
left=25mm,right=25mm,%
bindingoffset=0mm, top=20mm,bottom=20mm}


%\linespread{1.3}

\newcommand{\linia}{\rule{\linewidth}{0.5pt}}
\newcommand{\ano}{\text{1}}

% custom theorems if needed
\newtheoremstyle{mytheor}
    {1ex}{1ex}{\normalfont}{0pt}{\scshape}{.}{1ex}
    {{\thmname{#1 }}{\thmnumber{#2}}{\thmnote{ (#3)}}}

\theoremstyle{mytheor}
\newtheorem{defi}{Definition}

% my own titles
\makeatletter
\renewcommand{\maketitle}{
\begin{center}
\vspace{2ex}
{\huge \textsc{\@title}}
\vspace{1ex}
\\
ECN 102 - Digital Logic Design \\
Department of Electronics and Communication Engineering \\
Indian Institute of Technology, Roorkee
\linia\\
\@author \hfill \@date
\vspace{4ex}
\end{center}
}
\makeatother
%%%

% custom footers and headers
\usepackage{fancyhdr}
\pagestyle{fancy}
\lhead{}
\chead{}
\rhead{}
\lfoot{Assignment \ano - Introduction to Verilog}
\cfoot{}
\rfoot{Page \thepage}
\renewcommand{\headrulewidth}{0pt}
\renewcommand{\footrulewidth}{0pt}
%

\definecolor{vgreen}{RGB}{104,180,104}
\definecolor{vblue}{RGB}{49,49,255}
\definecolor{vorange}{RGB}{255,143,102}

\makeatletter
\newcommand*\@lbracket{[}
\newcommand*\@rbracket{]}
\newcommand*\@colon{:}
\newcommand*\colorIndex{%
    \edef\@temp{\the\lst@token}%
    \ifx\@temp\@lbracket \color{black}%
    \else\ifx\@temp\@rbracket \color{black}%
    \else\ifx\@temp\@colon \color{black}%
    \else \color{vorange}%
    \fi\fi\fi
}
\makeatother

\definecolor{codebg}{RGB}{250,250,240} 
\definecolor{greatblue}{RGB}{91,155,215} 

% Set up caption and labels for lstlistings
\DeclareCaptionFont{white}{\color{white}}
\DeclareCaptionFormat{listing}{\colorbox{greatblue}{\parbox{\textwidth}{\hspace{1cm}#1#2#3}}}
\captionsetup[lstlisting]{format=listing,labelfont=white,textfont=white}

\def\backtick{\char18} 
\lstdefinestyle{verilog-style}
{
    columns=fullflexible,
    language=Verilog,
    basicstyle=\small\ttfamily,
    keywordstyle=\color{vblue},
    identifierstyle=\color{black},
    commentstyle=\color{vgreen},
    numbers=left, 
    numberstyle=\color{gray},  
    numbersep=10pt,
    moredelim=*[s][\colorIndex]{[}{]},
    literate=*{:}{:}1, 
    backgroundcolor=\color{codebg},
    framexrightmargin=0.09cm, 
    framexleftmargin=-0.09cm,
    frame=trbl,
    upquote=true, 
    framerule=0pt
}

\newcommand{
  \insertverilog}[3]{
  \lstinputlisting[label=#2, caption=#3, style={verilog-style}]{#1}
}


%%%----------%%%----------%%%----------%%%----------%%%
% Command for creating a resource box
\newcommand{\resourcebox}[2]{
  \fbox{%
    \parbox{0.5\textwidth}{%
      \text{#1}
    }%
  } 
}


%%%----------%%%----------%%%----------%%%----------%%%


\begin{document}

\title{Assignment \ano \\ Introduction to Verilog}

\maketitle

\section*{Hardware Description Languages}
Hardware description language (HDL) is a convienient, device independent way of representing digital logic. HDLs are helpful describing, simulating and verifying digital circuits.

\subsection*{Why not use C/C++, Java...?} 
C/C++, Java are programming languages which are very good at what they were designed for, that is programming. However describing digital circuits in programming language is difficult and often confusing, which led to development of HDLs.

\subsection*{Which HDL to use?}
Today many Hardware description languages are available, each of them are different from each other in terms of functionality, semantic and grammar. In this course however we will be using Verilog  2001.

\section*{Using Verilog 2001}
As a begininer we are going to describe and simulate simple combinational circuit with Verilog.

\section*{Design Flow}  

\begin{figure}[!t] \centering 
  \includegraphics[width=\linewidth]{./resources/hdl_design_flow.pdf}
  \caption{Example binary search tree constructed using names from Table \ref{Table:ndn_name_table} with insertion order and output port information.}
  \label{Fig:bst_sample_names}
\end{figure} 

\section*{Verilog Syntax}
\subsection*{Modules}
Module is the basic building block in Verilog, modules helps in organizing and structuring designs in logical and human readable form. Module can be considered as analogus to functions in programming languages.

\insertverilog{./verilog_files/module.v}{sample-module}{\text{Sample module indicating its structure}}

An example of AND gate would be
\insertverilog{./verilog_files/andGate.v}{sample-module}{\text{Illustrative AND gate module}}

\subsection*{Instantiating modules}
The process of creating objects of modules is called instatiation in Verilog. Modules are a
  
\insertverilog{./verilog_files/AndGate3.v}{sample-module}{\text{Illustrative AND gate module}} 
\subsection*{Comments}
Comments in verilog are of two kinds:
\subsubsection*{Single Line Comment}
Two forward slashes represents begining of a single line comment in verilog, any charater after the two forward slashes will be ignored by verilog compiler until the end of line.
\insertverilog{./verilog_files/singleLineComment.v}{single-line-comment}{\text{Single line comment}} 

\subsubsection*{Block Comment}
Block comments in verilog are used to comment more than one line, the start with `/*' and ends with `*/'. Anything between these two character sequence is ignored by the compiler.
\insertverilog{./verilog_files/blockComment.v}{block-comment}{\text{Block
    comment}}

\subsection*{Numerical Literals}
\subsubsection*{Sized Numbers}
To represent digital circuits accurately Verilog allows defining
numbers of fixed size. These numbers have following format:
\begin{center}
  <size>'<character for base><number>
\end{center}

For example:
\insertverilog{./verilog_files/sizedNumbers.v}{sized-numbers}{\text{Example of sized numbers}}
 
\subsubsection*{Unisized Numbers}
Verilog also includes support for unsized numbers. These numbers are
assumed to be of a particular size depending upon the
compiler/machine.

\subsection*{Constants}
Global constants can be decalered in Verilog. When Verilog code is processed all these constants will be replaced by their respective values. NOTE: Verilog constants always starts with backtick ` ${}^{\backprime}$ '. 
\insertverilog{./verilog_files/constants.v}{constants}{\text{Declaration and use of constants}}

\subsection*{Wires}

\begin{figure}[!h] \centering  
  \includegraphics[width=\linewidth]{./resources/andGate4_representation.pdf}
  \caption{Gate level representation of code in Listing \ref{andGate4-wire}.} 
  \label{Fig:andGate4-representation}
\end{figure}

Wires are something very important in verilog, use of wires in Verilog is for what wires are used in real life ... to connect two things electrically! Wires will be used extensively in future assignments to connect two modules, registers and even wires together. This concept is explained using Listing \ref{constants} where wires are used to connect output of two 2-input AND gates to inputs of single 2-input AND gates. Gate level diagram of which is shown in Fig. \ref{Fig:andGate4-representation}.

\insertverilog{./verilog_files/andGate4.v}{andGate4-wire}{\text{Use of wires to connect output of one module to input of another}}


\subsection*{Registers}
\subsection*{Always Block} 
\subsection*{If and Case Statement}

\end{document}

% References:
% http://cva.stanford.edu/people/davidbbs/classes/ee108a/winter0607%20labs/ee108a_nham_intro_to_verilog.pdf
% Verilog HDL - Samir Palnitkar
