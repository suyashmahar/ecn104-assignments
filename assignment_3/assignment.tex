\documentclass[a4paper,10pt]{article}
   
\usepackage[T1]{fontenc}
\usepackage[utf8]{inputenc}
\usepackage{textcomp}
\usepackage[T1]{fontenc}
\usepackage{graphicx}
\usepackage{xcolor, colortbl}
\usepackage{caption}
\usepackage{listings}
\usepackage{wrapfig} 
\usepackage{tabu} % For coloring single row of table
\usepackage{scrextend}
%\usepackage{hyperref}

\renewcommand\familydefault{\sfdefault} 
\usepackage{tgheros}
\usepackage[defaultmono]{droidmono}

\usepackage{amsmath,amssymb,amsthm,textcomp}
\usepackage{enumerate}
\usepackage{multicol}
\usepackage{tikz}

\usepackage{geometry}
\usepackage{trace}
\usepackage{tcolorbox}
\usepackage{tabularx}
\usepackage{accsupp}% http://ctan.org/pkg/accsupp
\usepackage{enumitem}

%% For using numbers at every level of enumerated list
\newlist{legal}{enumerate}{10}
\setlist[legal]{label*=\arabic*.}

\tcbuselibrary{listings,skins} % For lstlisting

\geometry{total={210mm,297mm},
left=25mm,right=25mm,%
bindingoffset=0mm, top=20mm,bottom=20mm}


% For coloring single row in table
\def\zapcolorreset{\let\reset@color\relax\ignorespaces}
\def\colorrows#1{\noalign{\aftergroup\zapcolorreset#1}\ignorespaces}

\newcommand{\linia}{\rule{\linewidth}{0.5pt}}
\newcommand{\ano}{\text{3}}

% custom theorems if needed
\newtheoremstyle{mytheor}
    {1ex}{1ex}{\normalfont}{0pt}{\scshape}{.}{1ex}
    {{\thmname{#1 }}{\thmnumber{#2}}{\thmnote{ (#3)}}}

\theoremstyle{mytheor}
\newtheorem{defi}{Definition}

%%% Declare title %%%%%%%%%%%%%%%%%%%%%%%%%%%%%%%%%%%%%%%%%%%%%%%%
\newcommand{\antitle}{\text{Factorial Calculator using Verilog}}
% my own titles
\makeatletter
\renewcommand{\maketitle}{
\begin{center}
\vspace{2ex}
{\huge \textsc{{{\large}Assignment - \ano}\vspace{0.1cm} \break \antitle}}
\vspace{1ex}
\\
%%%%%%%%%%%%%%%%%%%%%%%%%%%%%%%%%%%%%%%%%%%%%%%%%%%%%%%%%%%%%%%%%%

Department of Electronics and Communication Engineering \\
Indian Institute of Technology, Roorkee
\linia\\
ECN 104 \hfill Digital Logic Design
\vspace{4ex}
\end{center}
}
\makeatother
%%%

% custom footers and headers
\usepackage{fancyhdr}
\pagestyle{fancy}
\lhead{}
\chead{}
\rhead{}
\lfoot{Assignment \ano\ - \antitle}
\cfoot{}
\rfoot{Page \thepage}
\renewcommand{\headrulewidth}{0pt}
\renewcommand{\footrulewidth}{0pt}
%

\definecolor{vgreen}{RGB}{104,180,104}
\definecolor{vblue}{RGB}{49,49,255}
\definecolor{vorange}{RGB}{255,143,102}

\makeatletter
\newcommand*\@lbracket{[}
\newcommand*\@rbracket{]}
\newcommand*\@colon{:}
\newcommand*\colorIndex{%
    \edef\@temp{\the\lst@token}%
    \ifx\@temp\@lbracket \color{black}%
    \else\ifx\@temp\@rbracket \color{black}%
    \else\ifx\@temp\@colon \color{black}%
    \else \color{vorange}%
    \fi\fi\fi
}
\makeatother

\definecolor{codebg}{RGB}{250,250,240} 
\definecolor{greatblue}{RGB}{91,155,215} 

% Set up caption and labels for lstlistings
\DeclareCaptionFont{white}{\color{white}}
\DeclareCaptionFormat{listing}{\colorbox{greatblue}{\parbox{\textwidth}{\hspace{1cm}#1#2#3}}}
\captionsetup[lstlisting]{format=listing,labelfont=white,textfont=white}

\renewcommand{\thelstnumber}{% Line number printing mechanism
  \protect\BeginAccSupp{ActualText={}}\arabic{lstnumber}\protect\EndAccSupp{}%
}

\def\backtick{\char18} 
\lstdefinestyle{verilog-style}
{
    %columns=fullflexible, 
    language=Verilog,
    basicstyle=\small\ttfamily,
    keywordstyle=\color{vblue},
    identifierstyle=\color{black},
    commentstyle=\color{vgreen},
    numbers=left, 
    numberstyle=\color{gray},  
    numbersep=10pt,
    moredelim=*[s][\colorIndex]{[}{]},
    literate=*{:}{:}1, 
%    backgroundcolor=\color{codebg},
%    framexrightmargin=0.09cm, 
    framexleftmargin=-0.09cm,
    frame=trbl,
    upquote=true, 
    framerule=0pt,
    keepspaces=true
}

\lstdefinestyle{verilog-inline-style}
{
    language=Verilog,
    basicstyle=\small\ttfamily,
    keywordstyle=\color{vblue},
    identifierstyle=\color{black},
    commentstyle=\color{vgreen},
    moredelim=*[s][\colorIndex]{[}{]},
    literate=*{:}{:}1, 
    upquote=true, 
    framerule=0pt,
    keepspaces=true
}

\newcommand{
  \insertverilog}[3]{
  \lstinputlisting[label=#2, caption=#3, style={verilog-style}]{#1}
}


% Command for problem
\newcounter{problemNumber}
\setcounter{problemNumber}{1}
\newcommand {
  \insertProblem}[1]{
  \vspace{0.5cm}
  \hrule
  \vspace{0.3cm}

  {\color{greatblue}\textbf{\large{Problem \theproblemNumber}}}
  \vspace{2pt}\\#1

  \addtocounter{problemNumber}{1}
  \vspace{0.2cm}
  \hrule  
  \vspace{0.5cm}
}


%%%----------%%%----------%%%----------%%%----------%%%
% Command for creating a resource box
\newcommand{\resourcebox}[2]{
  \fbox{%
    \parbox{0.5\textwidth}{%
      \text{#1}
    }%
  } 
}


%%%----------%%%----------%%%----------%%%----------%%%

\makeatletter
\def\lst@outputspace{{\ifx\lst@bkgcolor\empty\color{white}\else\lst@bkgcolor\fi\lst@visiblespace}}
\makeatother


%%%----------%%%----------%%%----------%%%----------%%%
\begin{document}

\title{Assignment \ano \\ Factorial Calculator}

\maketitle
\section*{Designing a factorial calculator}
This is a short assignment where you will be using the skills you have
already learned over the semester to design a simple factorial
calculator. The design specification for the calculator are following.

\begin{enumerate}
\item The calculator takes a single 32 bit number as input.
\item The input is unsigned, that is you don't have to take care of
  the sign (number can only be positive).
\item Output of the calculator includes a single 32 bit number which
  will be the result and a single bit output indicating the overflow.
\item Overflow occurs when more than 32 bits are required to represent
  the result.
\end{enumerate}

\subsection*{Hints}
\begin{enumerate}
\item You may use barrel shifter (designed in assignment 2) for
  performing multiplication.
\item Try not use `For loops' unless you are completely sure of what
  you are trying to do. `For loops' in Verilog are not for repeated
  execution of statements.
\item To change the radix of a signal in simulation window use the
  first few steps of this guide:\\
  https://sites.google.com/a/temple.edu/ece2612/home/xilinx-vivado-debugging
\end{enumerate}

\section*{Important guidlines for submission}
You should follow the following these guidelines carefully, failing to
which your assignment might not be properly evaluated.
\begin{enumerate}
\item You need to submit only one \textbf{ZIP} file; .RAR, .TAR or any
  other archive format will \textbf{\color{red}not} be accepted.
\item The ZIP file will only contain the following:
  \begin{enumerate}
  \item Source code of the factorial calculator.
  \item Testbench for the calculator.
  \item A \textbf{PDF} containing full screen screen-shots of your
    simulations, source code and elaborated design. Please note that
    the screenshots should clearly indicate the functioning of your
    module.\\ \textbf{\color{red}NOTE}: Word documents (*.docx), open
    document (*.odf) or any other format will {\color{red}not} be
    accepted.
  \item Name of the PDF should be your enrollment number,
    e.g. 171160XX.pdf.
  \item Simulation results should clearly show all the typical cases,
    including normal factorial calculation and overflow.
  \end{enumerate}
\end{enumerate}

\end{document}
