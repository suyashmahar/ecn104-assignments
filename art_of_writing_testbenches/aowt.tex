\documentclass[a4paper,10pt]{article}
 
\usepackage[T1]{fontenc}
\usepackage[utf8]{inputenc}
\usepackage{textcomp}
\usepackage[T1]{fontenc}
\usepackage{graphicx}
\usepackage{xcolor, colortbl}
\usepackage{caption}
\usepackage{listings}
\usepackage{wrapfig} 
\usepackage{tabu} % For coloring single row of table
\usepackage{scrextend}
%\usepackage{hyperref}

\renewcommand\familydefault{\sfdefault} 
\usepackage{tgheros}
\usepackage[defaultmono]{droidmono}

\usepackage{amsmath,amssymb,amsthm,textcomp}
\usepackage{enumerate}
\usepackage{multicol}
\usepackage{tikz}

\usepackage{geometry}
\usepackage{trace}
\usepackage{tcolorbox}
\usepackage{tabularx}
\usepackage{accsupp}% http://ctan.org/pkg/accsupp


\tcbuselibrary{listings,skins} % For lstlisting

\geometry{total={210mm,297mm},
left=25mm,right=25mm,%
bindingoffset=0mm, top=20mm,bottom=20mm}


% For coloring single row in table
\def\zapcolorreset{\let\reset@color\relax\ignorespaces}
\def\colorrows#1{\noalign{\aftergroup\zapcolorreset#1}\ignorespaces}

\newcommand{\linia}{\rule{\linewidth}{0.5pt}}
\newcommand{\ano}{\text{2}}

% custom theorems if needed
\newtheoremstyle{mytheor}
    {1ex}{1ex}{\normalfont}{0pt}{\scshape}{.}{1ex}
    {{\thmname{#1 }}{\thmnumber{#2}}{\thmnote{ (#3)}}}

\theoremstyle{mytheor}
\newtheorem{defi}{Definition}

%%% Declare title %%%%%%%%%%%%%%%%%%%%%%%%%%%%%%%%%%%%%%%%%%%%%%%%
\newcommand{\antitle}{\text{The Art of Writing Testbenches}}
% my own titles
\makeatletter
\renewcommand{\maketitle}{
\begin{center}
\vspace{2ex}
{\huge \textsc{\antitle}}
\vspace{1ex}
\\
%%%%%%%%%%%%%%%%%%%%%%%%%%%%%%%%%%%%%%%%%%%%%%%%%%%%%%%%%%%%%%%%%%

Department of Electronics and Communication Engineering \\
Indian Institute of Technology, Roorkee
\linia\\
ECN 104 \hfill Digital Logic Design
\vspace{4ex}
\end{center}
}
\makeatother
%%%

% custom footers and headers
\usepackage{fancyhdr}
\pagestyle{fancy}
\lhead{}
\chead{}
\rhead{}
\lfoot{\antitle}
\cfoot{}
\rfoot{Page \thepage}
\renewcommand{\headrulewidth}{0pt}
\renewcommand{\footrulewidth}{0pt}
%

\definecolor{vgreen}{RGB}{104,180,104}
\definecolor{vblue}{RGB}{49,49,255}
\definecolor{vorange}{RGB}{255,143,102}

\makeatletter
\newcommand*\@lbracket{[}
\newcommand*\@rbracket{]}
\newcommand*\@colon{:}
\newcommand*\colorIndex{%
    \edef\@temp{\the\lst@token}%
    \ifx\@temp\@lbracket \color{black}%
    \else\ifx\@temp\@rbracket \color{black}%
    \else\ifx\@temp\@colon \color{black}%
    \else \color{vorange}%
    \fi\fi\fi
}
\makeatother

\definecolor{codebg}{RGB}{250,250,240} 
\definecolor{greatblue}{RGB}{91,155,215} 

% Set up caption and labels for lstlistings
\DeclareCaptionFont{white}{\color{white}}
\DeclareCaptionFormat{listing}{\colorbox{greatblue}{\parbox{\textwidth}{\hspace{1cm}#1#2#3}}}
\captionsetup[lstlisting]{format=listing,labelfont=white,textfont=white}

\renewcommand{\thelstnumber}{% Line number printing mechanism
  \protect\BeginAccSupp{ActualText={}}\arabic{lstnumber}\protect\EndAccSupp{}%
}

\def\backtick{\char18} 
\lstdefinestyle{verilog-style}
{
    %columns=fullflexible, 
    language=Verilog,
    basicstyle=\small\ttfamily,
    keywordstyle=\color{vblue},
    identifierstyle=\color{black},
    commentstyle=\color{vgreen},
    numbers=left, 
    numberstyle=\color{gray},  
    numbersep=10pt,
    moredelim=*[s][\colorIndex]{[}{]},
    literate=*{:}{:}1, 
    %    backgroundcolor=\color{codebg},
    %    framexrightmargin=0.09cm,
    showstringspaces=false,
    framexleftmargin=-0.09cm,
    frame=trbl,
    upquote=true, 
    framerule=0pt,
    keepspaces=true
}

\lstdefinestyle{verilog-inline-style}
{
    language=Verilog,
    basicstyle=\small\ttfamily,
    keywordstyle=\color{vblue},
    identifierstyle=\color{black},
    commentstyle=\color{vgreen},
    moredelim=*[s][\colorIndex]{[}{]},
    literate=*{:}{:}1, 
    upquote=true, 
    framerule=0pt,
    keepspaces=true
}

\newcommand{
  \insertverilog}[3]{
  \lstinputlisting[label=#2, caption=#3, style={verilog-style}]{#1}
}


% Command for problem
\newcounter{problemNumber}
\setcounter{problemNumber}{1}
\newcommand {
  \insertProblem}[1]{
  \vspace{0.5cm}
  \hrule
  \vspace{0.3cm}

  {\color{greatblue}\textbf{\large{Problem \theproblemNumber}}}
  \vspace{2pt}\\#1

  \addtocounter{problemNumber}{1}
  \vspace{0.2cm}
  \hrule  
  \vspace{0.5cm}
}


%%%----------%%%----------%%%----------%%%----------%%%
% Command for creating a resource box
\newcommand{\resourcebox}[2]{
  \fbox{%
    \parbox{0.5\textwidth}{%
      \text{#1}
    }%
  } 
}


%%%----------%%%----------%%%----------%%%----------%%%

\makeatletter
\def\lst@outputspace{{\ifx\lst@bkgcolor\empty\color{white}\else\lst@bkgcolor\fi\lst@visiblespace}}
\makeatother


%%%----------%%%----------%%%----------%%%----------%%%
\begin{document}

\title{The Art of Writing Testbenches}

\maketitle
\tableofcontents

\section{Introduction}
In the first assignment we studied the basics of Verilog HDL, which we
extended in the second assignment and learnt about different modelling
techniques. But one thing might still bug you, which is you have
designed a module but how to verify it? This is where Verilog
testbenches are used, testbenches are verilog module that have sole
purpose of testing and verifying the design by simulating it, this can
significantly decrease the errors in the design.

\section{Types of Verilog Constructs}
\subsection{Synthesizable Verilog}
Anything written such that it can represent a valid circuit will be
synthesizable in Verilog, such code is used for designing modules.

\insertverilog
    {./verilog_files/synthCode.v}
    {synth-code-example}
    {\text{Example of synthesizable Verilog code.}}


\subsection{Non-synthesizable Verilog}
Modules written to verify other modules are non-synthesizable, such
modules cannot represent a valid digital circuit.

\insertverilog
  {./verilog_files/nonSynthCode.v}
  {non-synth-code-example}
  {\text{Example of non-synthesizable Verilog code (testbench\_1\_1.v from Assignment 1).}}


\subsection{Common Synthesizable and Non-Synthesizable Verilog constructs}
\begin{table}[h]
  \begin{center}
    \label{Table:operators-table}
    \caption{List of common operators used in Verilog}
    \renewcommand{\arraystretch}{1.1}
    \begin{tabularx}{\textwidth}{|X|X|}
      \hline
      \rowcolor{greatblue}
      \color{white}Synthesizable & \color{white}Non-synthesizable \\
      \hline
      n-Dimensional vectors & Ports having dimensionality greater\\
      (e.g. \lstinline[style=verilog-inline-style]{wire [1023:0] newWire [31:0];}) & than 1 \\
      \hline
      All operators & Delay statements \\
      \hline
      Part select & \\
      \hline
      If-else, case, casex and casez statements & \\
      \hline
      wires, regs & \\
      \hline
    \end{tabularx}
  \end{center}
\end{table}

\section{Basics of Writing a Testbench}
\subsection{The clock}
Every synchronous design has one or more clocks, these can be
initialized by declaring a reg which would hold the state of the clock
and changing it at the desired interval, an example of which is shown
in Listing \ref{clock-example}.

\insertverilog
    {./verilog_files/clockExample.v}
    {clock-example}
    {\text{Example clock of time period 15 ns.}}


Another and the preffered way to initialize a clock in Verilog is
using the forever block, which is described in Listing
\ref{clock-forever-example}.

\insertverilog
    {./verilog_files/clockForeverExample.v}
    {clock-forever-example}
    {\text{Example clock of time period 15 ns using forever statement.}}


\subsection{The I/O}
For complete testing of a design you'll have to simulate the inputs to
the module and verify its output. Inputs to a module are declared as
\lstinline[style=verilog-inline-style]{reg} while the output are
declared as \lstinline[style=verilog-inline-style]{wire}.

Inputs to a module are always changed from a procedural block, this is
why they are declared as \lstinline[style=verilog-inline-style]{reg},
while the outputs are just a connection from the output port of the
module under test to your testbench. It is important to note here that
output of a module can be left unconnected, but it is almost always a
good idea to verify every output of a module.

Listing \ref{io-in-testbench} shows how to declare I/Os of a module.
\insertverilog
    {./verilog_files/iosInTestbench.v}
    {io-in-testbench}
    {\text{Example code showing how to initialize and read I/Os of a module.}}

In the last example, all the initialization and stimulation was done
using \lstinline[style=verilog-inline-style]{initial} block, however
if stimulating inputs using initial blocks is getting cumbersome, you
can use \lstinline[style=verilog-inline-style]{always} block too. One
important point to note here is that
\lstinline[style=verilog-inline-style]{always} block in a testbench
starts executing simultaneously with all other
\lstinline[style=verilog-inline-style]{initial} and
\lstinline[style=verilog-inline-style]{always} blocks while
simulating. Example of using \lstinline[style=verilog-inline-style]{always} block for simulating a design is provided in Listing \ref{io-using-always}.

\insertverilog
    {./verilog_files/ioUsingAlways.v}
    {io-using-always}
    {\text{This listing shows how to use always block for simulating inputs to a module.}}

\end{document}

